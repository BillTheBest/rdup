\documentclass[a4, 10pt]{article}

\newcommand{\rdup}{\textbf{rdup}}
\newcommand{\cmd}[1]{\texttt{#1}}
\newcommand{\url}[1]{\texttt{#1}}

\textheight 210mm

\begin{document}
\pagestyle{empty}

\title{\rdup{} -- Easy (Personal) Backups}
\author{Miek Gieben\\miek@miek.nl}
\date{}
\maketitle
\thispagestyle{empty}

\paragraph{Abstract}
\rdup{} is small, GPL licensed utility which, together with some scripts, can
be easily used to setup and maintain backups. These can either be local
of networked backups. Encryption and compression is also supported.
The homepage of \rdup{} is located at: \\
\url{http://www.miek.nl/projects/rdup/index.html}.

\paragraph{Backups}
In an ideal world backups would not be necessary, with an unlimited
amount of memory and non failing hard disk and some form of version
management we would always be able to retrieve past versions of our files.
\begin{quote}
\emph{Unfortunately we do need backups.}
\end{quote}
Although the act of backing up may seem trivial (just copy some files
to another medium), a number of constraints make it more difficult. 
Backups may need to be compressed or encrypted. Backups might be copied
to another host. Backups may need to be split up in such a way that
each piece will fit on a CD or DVD. Incremental backups need to be
supported. And, one of the more important ones: recovery of backed up
files should be easy.  Also backup tools should be file system agnostic
and should be very flexible and simple in their use.

A backup tool should not re-invent the wheel.  This means that it should
try to use \cmd{ssh} for file transfers and not create a whole new
mechanism to do secure authentication. A program like \cmd{tar} can be
used to pack files, other tools can be used for encryption. 

Taking these requirements and thinking them through, the
conclusion is that the best backup program... does not backup anything.
Hence: \rdup

\paragraph{What is \rdup?}
\rdup{} is an utility inspired by \cmd{rsync} and the 
\cmd{plan9}\footnote{http://plan9.escet.urjc.es/magic/man2html/4/fs} 
way of doing backups. As said, \rdup{} does not create backups. It only
prints a list of files that are changed, or all files in case of a null
dump, to standard output. A sample of \rdup's output can be viewed in
Figure \ref{fig1:rdup}.

\begin{figure}[hbt]
\begin{verbatim}
+16877 1000 1000 11 0 /home/miekg
+16877 1000 1000 15 0 /home/miekg/svn
+16832 1000 1000 20 0 /home/miekg/svn/rdup
+16832 1000 1000 26 0 /home/miekg/svn/rdup/trunk
\end{verbatim}
\caption{Typical output from \rdup.}
\label{fig1:rdup}
\end{figure}

Subsequent programs in a shell pipe line can be used to actually
implement to backup scheme. A few scripts are included with \rdup, 
the most important ones are:
\begin{description}
        \item[\cmd{mirror.sh} --]{
                mirror/copies each file to the backup directory, remote 
                mirroring is also possible,}
        \item[\cmd{gzip.sh} --]{
                a compression filter,}
        \item[\cmd{crypt.sh} --]{
                an encryption filter.}
\end{description}

Also included is a \cmd{tar} based setup and an external (USB) drive
backup: "Turn on the drive and your backup starts."

\paragraph{Restoring}
Note that restoring files when mirrored is as easy as 'cp'-ing the
backed up file to the original location. Two shell scripts are
also provided inspect the backed up files:
\begin{description}
        \item[\cmd{hist.sh} --]{
                list the history of the backed up file, show the size
                and the backup date,}
        \item[\cmd{yesterday.sh} --]{
                generate a \cmd{diff}, show the file and optional copy 
                the file to another directory.}
\end{description}

Together these scripts provide a complete system
for creating and managing backups. When all backups are mirrored on a
remote system and the disk of that system is made read-only available to
the network, you have a complete network based backup system.

\paragraph{Examples of Backups with \rdup}
The most simple use of \rdup{} is to mirror files to
another partition. The files \cmd{timestamp} and \cmd{FILELIST} are
both created by \rdup:
\begin{quote}
\begin{verbatim}
rdup -N timestamp FILELIST /home | mirror.sh -b /vol/backup
\end{verbatim}
\end{quote}

\noindent If you want to remotely store your backups (note the \cmd{-c} switch which makes
\rdup{} include the files' content in its output):
\begin{quote}
\begin{verbatim}
rdup -c -N timestamp FILELIST /home | \
ssh miekg@example.nl mirror.sh -c -b /vol/backup
\end{verbatim}
\end{quote}

\noindent With \cmd{gzip.sh} files can be compressed:
\begin{quote}
\begin{verbatim}
rdup -c -N timestamp FILELIST /home | gzip.sh | \
ssh miekg@example.nl mirror.sh -c -b /vol/backup
\end{verbatim}
\end{quote}

\noindent And with \cmd{crypt.sh} files are encrypted:
\begin{quote}
\begin{verbatim}
rdup -c -N timestamp FILELIST /home | crypt.sh secret.file | \
ssh miekg@example.nl mirror.sh -c -b /vol/backup
\end{verbatim}
\end{quote}

\end{document}
