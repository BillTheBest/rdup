\documentclass{article}

\newcommand{\rdup}{\textbf{rdup}}
\newcommand{\cmd}[1]{\texttt{#1}}
\newcommand{\url}[1]{\texttt{#1}}

\begin{document}
\pagestyle{empty}

\title{\rdup{} -- Easy (personal) Backups}
\date{}
\maketitle

\section{Abstract}
GPL 
small
scripts implement backup 
The homepage of rdup is located at: 
\url{http://www.miek.nl/projects/rdup/index.html}



\section{Backups}
In an ideal world backups would not be necessary, with an unlimited
amount of memory and non failing hard disk and some form of version
management we would always be able to retrieve past versions of our files.
\paragraph{}
Unfortunately we do need backups.
\paragraph{}
Although the act of backing up may seems trivial (just copy some files
to another medium), a number of constraints make it more difficult. 
Backups may need to be compressed, or encrypted. Backups might be copied
to another hosts. Backups may need to be split up in such a way that
each piece will fit on a CD or DVD. Incremental backups need to be
supported. And, one of the more important ones: recovery of backed up
files should be easy.  Also backup tools should be file system agnostic
and should be very flexible in their use.

A lot of backup programs only provide a subset of these requirements or
focus on one or two items. Others will do everything and make setting up
and usage a pain.

A backup tool should not re-invent the wheel.  This means that it should
try to use \cmd{ssh} for file transfers and not create a whole new
mechanism to do secure authentication. A program like \cmd{tar} can be
used to pack files, other tools can be used for encryption. 

Taking these requirements and thinking them through, the
conclusion is that the best backup program... does not backup anything.
Hence: \rdup

\section{What is \rdup}
\rdup{} is an utility inspired by \cmd{rsync} and the 
plan9 \footnote{http://plan9.escet.urjc.es/magic/man2html/4/fs} 
way of doing backups. As said, \rdup{} does not create backups. It only
prints a list of files that are changed, or all files in case of a null
dump, to standard output.

Subsequent programs in a shell pipe line can be used to actually
implement to backup scheme. Currently implemented is a local and
remote mirroring capability (with optional compression and/or
encryption), a \cmd{tar} based setup (which is in turn based upon
hdup2 \footnote{\url{http://www.miek.nl/projects/hdup2/index.html}}), 
and a external (USB) drive backup. Turn on the drive and your backup starts.

Note that restoring files when mirrored is as easy as 'cp'-ing the
backed up file to the original location. Two shell scripts are
also provided inspect the backed up files. 
\begin{description}
        \item[\cmd{hist.sh} --]{
                list the history of the backed up file, show the size
                and the backup date,}
        \item[\cmd{yesterday.sh} --]{
                generate a \cmd{diff}, show the file and optional copy 
                the file to another directory.}
\end{description}

Together with \cmd{mirror.sh} these scripts provide a complete system
for making and managing backups. When all backups are mirrored on a
remote system and the disk of that system is made read-only available on 
the network, you have a complete network based backup system.

\section{Examples of Backups with rdup}
The most simple use of \rdup is to mirror your files to
another partition. The following pipe line creates a
full dump:
\begin{quote}
\begin{verbatim}
rdup -N timestamp HOMELIST ~ | mirror.sh -b /vol/backup
\end{verbatim}
\end{quote}

\end{document}
