\documentclass[a4paper, openany]{blocksbook}
\newcommand{\rdup}{\textbf{rdup}}
\newcommand{\cmd}[1]{\texttt{#1}}
\newcommand{\url}[1]{\texttt{#1}}
\newcommand{\path}[1]{\texttt{#1}}
\newcommand{\flag}[1]{\textit{#1}}
\begin{document}

\chapter*{\rdup: Quick Start}
This is a small guide about setting up and running \rdup{} to make
backups. \rdup{}
supports incremental, compressed, encrypted, local, remote  backups and
restores. All these feature can (and or) implemented in scripts to
implement an actual backup scheme. The core \rdup{} utility does
nothing but to print out the files names that are to be backed up.

See \url{www.miek.nl/projects/rdup} for more information.

\subsection*{\textit{Setup}}
Get \rdup{} from its website. Configure and install it:
\begin{verbatim}
./configure && make && sudo make install
\end{verbatim}
\rdup{} by default creates a full dump once a month and 
daily incrementals after that (if desired). But this is
all in the hands of the user.

\subsection*{\textit{Backup}}
The most convenient way to use \rdup{} is to use the wrapper
script \cmd{rdup-dump}. Synopsis:\\
\cmd{rdup-dump [ OPTIONS ] DIR|FILE [ DIR|FILE ]}
\begin{itemize}
\item
Dump \path{/home} to your USB disk on \path{/media/usbdisk}:\\
\cmd{rdup-dump -e -b /media/usbdisk/\$HOST /home}
\item
Dump compressed \path{/home} to \path{/vol/backup}:\\
\cmd{rdup-dump -z b /vol/backup/\$HOST /home} 
\item
Dump encrypted \path{/home} to \path{/vol/backup} at remote
host:\\
\cmd{rdup-dump -k secret\_file -b /vol/backup/\$HOME \\\
-c user@example.nl /home}
\end{itemize}

\subsection*{\textit{Restore}}
The most convenient way to restore with \rdup{} is by using
the script \cmd{rdup-restore}. Synopsis:\\
\cmd{rdup-restore [ OPTIONS ] -b TO\_DIR FROM\_DIR [ FROM\_DIR ]}
\begin{itemize}
\item
Restore from the latest backup from May 2006, to \path{/tmp/restore}:\\
\cmd{rdup-restore -b /tmp/restore \\\
/vol/backup/elektron/200605/home/miekg}
\item
Restore from the latest remote backup from May 2006. This backup
is encrypted:\\
\cmd{rdup-restore -k secret\_file -c user@example.nl -b /tmp/restore \\\
/vol/backup/elektron/200605/home/miekg}
\end{itemize}

\end{document}
